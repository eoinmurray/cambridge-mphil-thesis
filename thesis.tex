
\documentclass[12pt, oneside]{book}
\usepackage[tt]{titlepic}
\usepackage{fancyhdr}
\usepackage[pdftex]{graphicx}
\usepackage[small,bf]{caption}
\usepackage{amsmath}
\usepackage{subcaption}
\usepackage[section]{placeins}
\usepackage{afterpage}

\usepackage[top=1in, bottom=1in, inner=1in,outer=1in]{geometry}
\usepackage{import}
\usepackage{dcolumn}
\usepackage{bm}
\usepackage{subcaption}

\usepackage{subfig}
\usepackage{cleveref}
\captionsetup[subfigure]{subrefformat=simple,labelformat=simple}

\linespread{1.5} \setlength{\parskip}{\baselineskip}%
\setlength{\parindent}{0pt}%%%%<---

\begin{document}

\title{} \titlepic{\subimport{title/}{titlepage.tex}} \author{} \date{}
\maketitle

\begin{center} \large \textbf{Acknowledgements} \end{center}

This project was very much a team effort and I would first like to thank all the
members of the integration project team. Thanks to David Ellis for designing and
fabricating the devices, the devices were fabricated at the Cavendish and the
e-beam lithography was undertaken by Jonathan Griffiths. Thanks to Thomas Meany
for packaging the hybrid device and taking measurements, Frederick Floether for
designing the devices, Jamie Lee for assisting with measurements and to Anthony
Bennet for designing and taking measurements. Thanks also for Ian Farrer and the
staff at the Cavendish for growing wafers. Thanks to Andrew Shields and David
Ritchie for giving me this opportunity to study for this PhD. This project was
supported by the Initial Training Network PICQUE under the European Commission
Marie Curie Actions.

\newpage \begin{center} \large \textbf{Abstract} \end{center}

Fundamental to the future of quantum photonics is the ability to create
integrated devices. An integrated chip offers intrinsic stability and
compactness. This makes it inherently more scalable than a bulk optics approach.
Quantum dots (QDs) are a developing on-chip source of single photons. This
project aims to take quantum dots embedded in a III-V material and combine them
with integrated waveguide components. The InAs QDs are grown in GaAs substrate
and are bonded to a SiON based waveguide platform. This approach has potential
for realization of an efficient coupling between the quantum dots and the
waveguides and then on-chip manipulation of the emitted photons. In this report,
the results thus far will be presented and proposals for future designs will be
discussed.

\newpage \large\textbf{Publications resulting from this work}

\small \textbf{Quantum photonics hybrid integration platform} \newline
\textbf{E. Murray}, D. J. P. Ellis, T. Meany, F. F. Floether, J. P. Lee, J. P.
Griffiths, G. A. C. Jones, I. Farrer, D. A. Ritchie, A. J. Bennett, and A. J.
Shields. \newline Applied Physics Letters 107.17 (2015): 171108.

\small \textbf{Cavity-enhanced coherent light scattering from a quantum dot} A.
J. Bennett, J. P. Lee, D. J. P. Ellis, T. Meany, \textbf{E. Murray}, F. F.
Floether, J. P. Griffiths, I. Farrer, D. A. Ritchie, and A. J. Shields. \newline
arXiv pre-print: 1508.01637v1

\large\textbf{Conferences}

\small \textbf{Quantum dots for quantum information science}, \newline
\textbf{E. Murray} \newline PICQUE Integrated Quantum Photonics Workshop, 7-9
January 2015, Oxford, UK

\small \textbf{Quantum photonics hybrid integration platform} \newline
\textbf{E. Murray}, D. J. P. Ellis, T. Meany, F. F. Floether, J. P. Lee, J. P.
Griffiths, G. A. C. Jones, I. Farrer, D. A. Ritchie, A. J. Bennett, and A. J.
Shields. \newline \textbf{Best talk awarded} PICQUE Roma Scientific School, 6-10
July 2015, Rome, Italy

\newpage \tableofcontents

\subimport{chapter_1/}{chap1.tex}
%\subimport{chapter_2/}{chap2.tex}
%\subimport{chapter_3/}{chap3.tex}
%\subimport{chapter_4/}{chap4.tex}

\bibliographystyle{unsrt} \bibliography{bibliography}{}

\end{document}
