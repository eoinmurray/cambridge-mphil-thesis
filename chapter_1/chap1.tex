
  \chapter{Introduction and theory.}

Linear optical quantum computing (LOQC) has been proven to be computationally
efficient with a single photon source and a series of beamsplitters and phase
shifters \cite{knill2001scheme}. Some implementations of two photon gates have
been realised with bulk optics \cite{o2003demonstration}, however a scalable
LOQC is unfeasible with bulk optics and an integrated technology is needed
\cite{carolan2015universal}. Semiconductor waveguides with integrated quantum
dots (QD) are a promising solution. Semiconductor III-V QDs have been shown to
have good, bright, single photon emission \cite{Bennett:05}, can emit
indistinguishable and entangled photons \cite{he2013demand,stevenson2012}, can
be site-controlled \cite{juska2013towards} and are compatible with semiconductor
foundry techniques. Progress is being made to embed the QDs into integrated
waveguide platforms. Integrated photonics offers the potential for true
scalability due to component miniaturisation. Stability is intrinsic to the
platform and offers a reduction in complexity and size of the device
\cite{politi2009integrated}. Many of the elements needed for LOQC can be
manufactured on-chip, high fidelity beam splitters and Mach Zehnder
interferometers (MZIs) can be made in various semiconductors
\cite{wang2014gallium, zhang2011, politi2008silica} as well as on-chip detectors
\cite{gerrits2011chip, hadfield2009single} however further work is needed to
integrate these components on with a quantum light source.

\section{Introduction to Quantum dots}

Semiconductor quantum dots (QDs) are islands of a certain material
surrounded by a material of a higher band gap. The QD region is a three
dimensional structure which confines carriers in all dimensions, giving carries freedom to move in zero dimensions. This
confinement gives rise to discrete energy levels inside the QD.

Normally a QD will confine two electron levels and two hole levels, more may be
accommodated in a larger QD. Each level will only confine two carriers due to the
Pauli exclusion principle. Figure \ref{fig:estructure} shows the electronic
structure of a QD. An electron excited to the conduction band leave a hole in
the valence band and is referred to as an e-h pair. An e-h pair confined in the
QD is referred to as an exciton, two e-h pairs is a biexciton. When there is an
imbalance in the number of electrons and holes a charged exciton is created.
Electron-hole pairs can be excited into the quantum dot by numerous methods -
primarily electro and photo-excitation. In the non-resonant case, carriers are
created in the vicinity of the QD by an above band excitation laser or an
electric current. These carriers will relax into the available quantum
potentials, the lowest energy of which is typically the QD.

\begin{figure}[h!] \begin{center}
\includegraphics[width=1\textwidth]{images/estructure.pdf} \end{center}
\caption{
Electronic structure of a quantum dot. (a) An e-h pair is referred to as an
exciton. (b) Two e-h pairs is referred to as a biexciton. (c) An exciton with an
extra hole (electron) is a positively (negatively) charged exciton.
} \label{fig:estructure} \end{figure}

The discrete energy levels in the QD give rise to clean, sharp emission lines.
There is a well defined energy for each transition and as such the emitted
photons have a well defined wavelength, limited by the lifetime of the state.

\subsection{Single photon emission from Quantum dots }

The angular momentum of an electron is $\pm 1/2$ and for a hole is $\pm 3/2$.
The e-h pair can combine to make a state with spin $\pm 1$ for photon emission
and $\pm 2$ for non-radiative recombination. The energy of the emitted photon
corresponds to the energy of its original excitonic state. Thus the emitted
photons can be spectrally filtered to analyse the emission from only one
excitonic state.

\begin{figure}[h!] \begin{center}
\includegraphics[width=1\textwidth]{images/qd_photon.pdf} \end{center}
\caption{ A single photon emission process in a quantum dot. A photon ($\gamma$ in the
image) excites an electron from the Valence band to the Conduction band, this
creates an exciton. This exciton can relax into the quantum dot and after some
time $\tau$ it will recombine and emit another photon. Since the quantum dot
energy leves are discrete, and only accept a fixed number of carriers and after
photon emission there is a finite time which must pass before more carriers can
be captured, true single photon emission is possible.
} \label{fig:qd_photon}
\end{figure}

For a light source with Poissonian emission statistics such as a laser, true
single photon emission can not be achieved even with significant attenuation
because the probability for multi photon emission still exists. In the case of a
quantum dot, since each excitonic energy level can only accept a fixed number of
carriers and after photon emission there is a finite time which must pass before
more carriers can be captured, true single photon emission is possible. This
process is shown schematically in Figure \ref{fig:qd_photon}. This single photon
emission can be verified by measuring the two photon second order correlation
function \begin{equation} g^{(2)}(\tau) = \frac{\left\langle I(t)I(t+\tau)
\right\rangle}{\left\langle I(t) \right\rangle \left\langle I(t+\tau)
\right\rangle} \end{equation} This is achieved by doing a Hanbury Brown Twiss
measurement \cite{brown1956test}. A stream of single photons impinge on a 50:50 beamsplitter. The two
outputs of the beamsplitters are sent to single photon detectors. One of the
detector signals is delayed by a time $\tau$ in order to measure both positive
and negative correlation times. If the source is a true single photon source
there will be a lack of coincidence clicks on both detectors when $\tau = 0$,
and thus $g^{(2)}(0)$ will be zero. Single photon emission from QDs has been
demonstrated under a wide range of excitation conditions: electrical (DC and AC)
and optical (continuous wave (cw) and pulsed lasers at above band,
quasi-resonant and resonant regimes).

\section{Semiconductor waveguides}

A dielectric waveguide consists of a core material surrounded by a second material of lower
refractive index. The surrounding material is known as the waveguide cladding.
Here the theory is set out as in Integrated Photonics by Saleh and Teich
\cite{saleh1991fundamentals}. As an introduction to the design of waveguides in
this section the prorogation of light in a symmetric dielectric slab waveguide
will be discussed. A core slab of refractive index $n_1$ is clad by two slabs of
refractive index $n_2$. This structure is shown in Figure
\ref{fig:planar_reflection}. The critical angle for the total internal
reflection of light is derived from Snells law to be

\begin{equation} \theta_c = \arcsin{\frac{n_1}{n_2}}. \end{equation}

When light impinges on the boundary of $n_1$ and $n_2$ with an angle less than
$\theta_c$ then it is reflected. Figure \ref{fig:planar_reflection} shows a
guided and an unguided ray of light. The unguided ray hits the boundary with an
angle greater than $\theta_c$ and therefore some of the light refracts,
losing some of the light at each reflection causing the ray to
eventually vanish. The guided ray hits the boundary at an angle less than
$\theta_c$ and reflects without any loss of power and will propagate along the
core.

\begin{figure}[h!] \begin{center}
\includegraphics[width=0.8\textwidth]{images/thesis_planar_reflection.pdf}
\end{center} \caption{
Light ray propagation in a planar dielectric waveguide. The unguided ray
impinges on the boundary at angle larger than $\theta_c$ and therefore some of
the light refracts, losing power at each reflection causing the ray to gradually
disappear. The guided ray impinges at an angle less than $\theta_c$ and reflects
fully propagating long the core without any loss of power.
}
\label{fig:planar_reflection} \end{figure}

To understand the waveguide modes in the structure assume there is a transverse
electromagnetic (TEM) plane wave propagating in the waveguide core. This TEM
wave has a wavelength $\lambda = \lambda_0/n_1$ where $\lambda_0$ is the free
space wavelength. The wave is reflected each time at the boundary with an angle
less than $\theta_c$ so the wave propagates in the core without loss of power.
With each reflection the wave lags behind the original by a distance
$2d\sin{\theta}$ \cite{saleh1991fundamentals}. There is also a phase $\phi_r$
change induced by each internal reflection. There is now a self-consistency
condition imposed upon the reflected wave. As the wave reflects twice it
reproduces itself, waves which satisfy this condition are known as eigenmodes,
or modes of the waveguide. The wave interferes with itself and a pattern is created
which does not change with $\hat{z}$. Due to this self-consistency the phase
shift between the waves is zero or a multiple of $2\pi$, giving rise to the
condition

\begin{equation}\label{eqn:mode_eqn} 2 k_y d - 2 \phi_r = 2\pi m, \ \ \ \ m =
\mathrm{0, 1, 2,...} \end{equation}

where $k_y = n_1 k_0 \sin{\theta}$. The phase shift $\phi_r$ in the transverse
electric (TE) case due to total internal reflection is given by

\begin{equation} \tan{\frac{\phi_r}{2}} =
\sqrt{\frac{\sin^2\theta_c}{\sin^2\theta} - 1}. \end{equation}

Combining this with Equation \ref{eqn:mode_eqn} it is deduced that

\begin{equation}\label{eqn:trans} \tan \left( \pi \frac{d}{\lambda} \sin \theta -
m \frac{\pi}{2} \right) = \sqrt{\frac{\sin^2\theta_c}{\sin^2\theta} - 1}
\end{equation}

\begin{figure}[h!] \begin{center}
\includegraphics[width=0.8\textwidth]{images/mode.pdf} \end{center}
\caption{Graphical solution of to the transcendental equation \ref{eqn:trans}. The black line is the
LHS of the equation and the red line is the RHS. Each LHS segment corresponds to
a mode in the waveguide.} \label{fig:mode} \end{figure}

This equation is transcendental and can be solved graphically, this solution is
shown in Figure \ref{fig:mode}. The LHS is a set of tan, when $m$ is even, and
cot functions, when $m$ is odd. Each LHS segment corresponds to a mode in the
waveguide. When the LHS crosses the x-axis, indicated by the semicircles, $\sin
\theta_m$ can be determined. The distance between these intersections is
$\lambda/2d$. The solution for transverse magnetic (TM) is very similar, with
just a different phase change upon reflection $\phi_r$. This information can be
used to deduce how many TE modes are in a waveguide of a given size $d$. The
number of modes M is given by the amount of segments of width $\lambda/2d$ exist
before $\sin \theta$ reaches the value $\sin \theta_c$.

\begin{equation}\label{eqn:nummodes} M = \frac{\sin \theta_c}{ \lambda / 2d},
\end{equation}

rounded up to the nearest integer. By substituting $\cos \theta_c = n_2/n_1$
into Equation \ref{eqn:nummodes} it is seen that

\begin{equation}\label{eqn:nummodes} M = 2\frac{d}{\lambda} \mathrm{NA}
\end{equation}

where

\begin{equation} \mathrm{NA} = \sqrt{n_1^2-n_2^2} \end{equation} is the
numerical aperture (NA) of the waveguide. The $\mathrm{NA}$ defines the range of
angle from which the waveguide can collect light. Equation \label{eqn:nummodes}
is extremely important in waveguide design because it defines the single-mode
cutoff for when $M \leq 1$. This impacts the choice of $d$ and the wavelength
for which the waveguide is single mode or multimode. Single mode waveguides are
generally more useful for quantum optics because they allow two photon
interference between light coming in from two single mode channels, this allows
the operation of directional couplers and Mach Zehnder interferometers. Single
mode optical fibres have lower loss in the telecommunications wavelength ranges
and so is better suited to long distance communication.

\subsection{Two dimensional waveguides}

\begin{figure}[h!] \begin{center}
\includegraphics[width=0.5\textwidth]{images/2d_waveguide.pdf} \end{center}
\caption{Index structure of a waveguide which confines light in two dimensions.
The light is confined in $\hat{x}$ and $\hat{y}$ but is free to propagate in the
$\hat{z}$ direction.} \label{fig:2dwg} \end{figure}

A two dimensional dielectric waveguide is one which confines light along two
axes and allows the light to propagate freely along the third axis. A schematic
of a rectangular waveguide is shown in Figure \ref{fig:2dwg}. The modes and
number of modes can be derived in a similar fashion as for the planar dielectric
waveguide and shall just be quoted here \cite{saleh1991fundamentals}. The number
of modes in a two dimensional waveguide can be approximated by

\begin{equation} M \approx \frac{\pi}{4} \left( \frac{2d}{\lambda_o } \right)^2
\mathrm{NA}^2. \end{equation}

This approximation holds well for multimode waveguides, but loses accuracy near
the single mode boundary. Design of single mode two dimensional waveguides
generally needs to be solved computationally.

\subsection{Directional couplers and Mach Zehnder interferometers}

\begin{figure}[h!] \begin{center}
\includegraphics[width=0.7\textwidth]{images/wg_devices.pdf} \end{center}
\caption{Two common waveguide devices, a directional coupler and a Mach Zehnder
interferometer. The numbers on the directional couplers indicate input and
output ports which shall be reference in the text, equivalent ports shall be
used for the Mach Zehnder interferometer.} \label{fig:wg_devices} \end{figure}

Figure \ref{fig:wg_devices} shows the schematic of two waveguide devices used
frequently throughout this project. A directional coupler (DC) is a photonic
device which splits the power between two output ports with a well defined
coupling ratio between the ports. Two waveguides are brought close enough so
that the evanescent fields overlap. The waveguides are assumed to be single
mode. It can be derived that the power in each port 3 and 4 in Figure
\ref{fig:wg_devices} as a result of light of power $P_1$ being injected to the
coupler from port 1 is

\begin{equation} P_3(z) = P_1 \cos^2 cz \end{equation} \begin{equation} P_4(z) =
P_1 \sin^2 cz \end{equation}

Where $z$ is the propagation direction and $c$ is the coupling coefficient
related to the index contrast of the waveguide, the separation $d$ between the
waveguides in the interaction region and the operating wavelength.

A Mach Zehnder interferometer (MZI) is a sequence of two DCs in series. The
light splits 50/50 into two arms by the first DC. On one of the arms is a local
refractive index changing element, this can be a electrode to take advantage of
the electro-optic effect or heater, for the thermo-optic effect. Here it shall
be assumed to be a heater. This element changes the local refractive index in
the arms. This causes a phase mismatch between the light propagating in each arm
so that the light will constructively or destructively interfere at the second
DC. By tuning the heater this phase can be chosen an light can be arbitrarily
coupled between ports 3 and 4. This is called a Mach Zehnder interferometer
switch or modulator. These are used extensively in telecommunications to route
signals between different channels.

 \section{Integrated quantum devices}

Semiconductor quantum dots and waveguides are interesting technologies on their
own. However this thesis focuses on the combination of these technologies to
create a platform capable of making working integrated quantum devices in the
future.

In Chapter 2 and 3, we attach a QD
filled GaAs chip to the end of a silicon oxynitride (hereafter SiON) based photonic circuit. This approach
allows us to make custom designed photonic circuits, which are fabricated
separately from the QD quantum light source. The circuits are made up of a
series of MZI's. This thesis focuses on the initial fabrication and
demonstration of the platform. In the next phase of research there are many
target circuits which are interesting for quantum telecommunications which have
been demonstrated with QD's with bulk optics, but not yet reproduced on an
integrated device. Theses include CNOT (controlled NOT) gates\cite{pooley2012controlled}, quantum relays'
\cite{varnava2015entangled}, quantum amplifiers \cite{kocsis2013heralded,
zavatta2011high}, NOON state generation \cite{bennett2015cavity, afek2010high, giovannetti2011advances} and various
combiners and splitters.

\subsection{NOON states and quantum sensing}

This section will describe a useful application of quantum sources and quantum
circuits: generating a NOON state. The dynamics of a NOON state as it passes
through a MZI allow it to be super sensitive to phase changes in the MZI \cite{afek2010high, PhysRevLett.107.083601}. A NOON
state is a many particle quantum entangled state represented as:

\begin{equation} \left|\psi\right\rangle = \left|N\right\rangle_{a}
\left|0\right\rangle_{b} + e^{iN\theta} \left|0\right\rangle_{a}
\left|N\right\rangle_{b} \end{equation}

this state is a superposition of $N$ particles in mode $a$ and 0 particles in
mode $b$ and vica-versa. The particles need to be bosonic and in this case the
quantum particles are photons.

To generate this state using photonics we consider the state generated by
photons propagating through an MZI. In the single photon case, single-photon
interference occurs from the path length difference in the two arms of the MZI.
One of the photon modes gathers a phase shift $\Delta \phi$ and causes the
detection probabilities of the MZI output arms to change.

The probabilities become $P_1 = 1 + \cos{\Delta \phi}$ and $P_2 = 1 +
\sin{\Delta \phi}$.

where the subscript indicates the path mode as in Fig \ref{fig:wg_devices}. This interference is generalised by Hong-Ou-Mandel interference
\cite{hong1987measurement} for multiple photons. The biphoton state must then be
represented by a path entangled state:

\begin{equation} \left|\psi\right\rangle = \left|2\right\rangle_{1}
\left|0\right\rangle_{2} + e^{i2\theta} \left|0\right\rangle_{1}
\left|2\right\rangle_{2} \end{equation}

In this case the oscillation of the detection probabilities depending on phase
is twice as fast. $P_{1} = 1 + \cos{2 \Delta \phi}$ and $P_{2} = 1 +
\sin{2 \Delta \phi}$.

This path encoded biphoton state is the same a two particle NOON state.

It is seen that the phase modulation increases linearly with the number of
particles in the state $N\Delta \phi$. This would allow smaller phase changes to
be detected faster by some quantum sensor, this is known as superresolution.
It can also be seen that the error in the phase measurement becomes smaller for
larger $N$. Consider the observable

\begin{equation} O = \left|N, 0\right\rangle \left\langle 0, N\right| + \left|0,
N\right\rangle \left\langle N, 0\right| \end{equation}

The error in the phase becomes

\begin{equation} \Delta \phi = \frac{\Delta O}{ | d \left\langle O \right\rangle /
d \phi | } = \frac{1}{N} \end{equation}

Thus the phase error decreases with increasing particle number, this is known as
supersensitivity.

\begin{figure}[h!] \begin{center}
\includegraphics[width=0.8\textwidth]{images/noon.pdf} \end{center}
\caption{
A single photon is emitted and impinges on the 50:50 beamsplitter, where one is
delayed. The two photons will then interfere at the first DC, generating a NOON
state. The state travels down paths 1 and 2, gathering a phase due to the
presence of the heater on path 1. The phase of the MZI, and thus the temperature
of the heater, can be determined accurately from analysing how output
coincidences change with temperature.
} \label{fig:noon-chip} \end{figure}

NOON state super resolution can be achieved experimentally using quantum dots
and an integrated photonic circuit \cite{bennett2015cavity}, shown schematically
here in Fig \ref{fig:noon-chip}.
The photons are generated sequentially from the QD, they are directed to a 50:50
beamsplitter where one is delayed. This causes 25\% of the sequential photons
to be time synched in parallel. QDs are resonantly excited in order to achieve
long coherence times and good two photon interference visibilities. The two photons
are then sent into a MZI, where they will interfere at the first DC, and generate a
NOON state whereby detecting the coincidences. This NOON state then travels down paths
1 and 2. Gathering a phase because of the presence of the heater on path 1.
from the outputs the phase of the MZI can be determined accurately.

In a photonic circuit, the phase of one arm is changed by inducing a
refractive index  change on that arm. This refractive index change causes a
change in the phase of  the circuit. This phase change is then measured with
high resolution by the NOON state, allowing a superresolution measurement of
whatever caused the refractive index change. Any component which changes the
refractive index of one arm can then be sensed.

For example, in the circuit shown in Fig \ref{fig:noon-chip},  placing a
conducting gold strip over one arm will induce a refractive index change when
there is a current run through the strip. This current can then be
superresolved. If a microfluidic channel is placed in the path of the light
going down one arm of the MZI, the fluid will have a different refractive index
depending on its contents. This can be used to sense trace amounts of components
inside the fluid \cite{crespi2012measuring}.

In this example\cite{bennett2015cavity}, the quantum light source and the photonic chip were not integrated. Meaning
that experimentally this method is fragile and required a lot of manual alignment. If the QD source
was bonded directly to the photonic circuit, and greater stability would be achieved, and also
the potential to use this technique in real applications, not just in a lab.

\section{Conclusion}
This chapter contains an introduction to the theory of both quantum dots and
photonic circuits. It also lays out the motivation for combining these
technologies.

The rest of this thesis discusses a hybrid method for combining a quantum source
with a photonic circuit. Chapter 2 discusses the theoretical and experimental
methods used in designing and fabricating the hybrid device. Chapter 3 outlines
the achieved results, a first demonstration of QDs integrated with photonic
circuits with an active manipulation element. Chapter 4 discusses promising
paths for future work.
