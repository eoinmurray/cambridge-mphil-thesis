
\chapter{Future work}

This document outlines work undertaken to design and fabricate an integrated
device which emits and modulates single photon emission. This device has great
potential in the field of quantum information science however some challenges
still need to be over come.

The device efficiency was calculated to be 2.8\%. This needs to improve to make
a practical device. The orthogonal bonding technique of
the hybrid device however has the advantage that the surface emission from a
III-V chip is easily optimised. The planar cavity used so far in this project
emits most of the light into the plane of the cavity and not to the surface of
the device. This can be negated by creating some kind of micropillar or nanowire
device where the mode directs all of the emission to the surface. The challenges
with this approach would be how to align the micropillars with the waveguides.
Another approach would be to change the index contrast of the waveguides.
Currently the cladding index of 1.51 and the SiON core index is 1.55, this core
index can be changed by varying the nitrogen concentration and could allow core
indices upto 2. This would increase the NA of the waveguide and collect more of
the light emitted from the surface of the III-V. A better mode matching between
the waveguides and optical fibres could be achieved by tapering the ends of the
waveguides and/or by using fibres with a lensed ends which alter the output mode
size.

Other important aspect of the device which need to be addressed is the
alignment of the III-V source. For a device which performs real operations many
QD photons need to interact on the chip. Many QDs need to be aligned with many waveguides.
This could be done using positioned QDs.

In order to make many photons interfere on-chip the QDs need to be wavelengths
tunable.  This could be achieved using QDs embedded in an electrically
controlled diode. This would allow the application of an electric field to the
QD to change the band structure and the emission energy of the photons. This
would allow many photons at the same wavelength propagate through the chip where
real logic operations could take place.
